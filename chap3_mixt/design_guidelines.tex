%!TEX root = ../thesis.tex
\subsection{Design Guidelines}

Based on the findings from our formative study, we propose four design guidelines for creating effective mixed media tutorials that combine text, images and videos.

Scannable steps: Scannable steps provide valuable context and facilitate navigation within tutorials. To leverage these benefits, videos in mixed tutorials should be presented in a format that supports scanning.

Small but legible videos: To make mixed media tutorials scannable and enable users to work with the tutorial and their application side-by-side, the videos for individual steps should use the minimum amount of screen real estate while still being legible. Ideally, videos should clearly depict the most important portions of the UI for each step (e.g., dialog boxes, panels, or canvas) while hiding or deemphasizing less relevant regions.

Visualize mouse movement: Our study indicates that videos are most useful for steps that involve brushing, drawing and manipulating control points, but even in videos, it can be difficult to see the exact motion or path of the mouse during such interactions. Visualizing mouse movement and events helps viewers understand the relevant spatio-temporal characteristics of the demonstration.

Give control to the user: Our observations of user behavior suggest that expertise and familiarity with the specific tools or interactions in a tutorial is likely to have an impact on which instructional format (static or video) is best for a given user. Videos help users understand, confirm and debug steps with unfamiliar tools, while static images and text are quicker and easier to skim. Thus, mixed tutorials should let users choose the most appropriate format at the granularity of individual steps.
