%!TEX root = ../thesis.tex
\section{Introduction}

Learning how to use software applications often happens opportunistically as users need to accomplish specific tasks. When it is unclear how to achieve the desired results, many users turn to step-by-step tutorials, which describe the set of operations required to complete a task. Visual editing applications, such as applications for drawing, photo editing, and 3D modeling, require visual tutorials that show not only how to navigate the user interface but also how to manipulate the canvas, image, or 3D model.

There are two main forms of visual step-by-step tutorials. Static tutorials use text and images to describe the set of operations required to accomplish a task. Video tutorials are screen recordings of the tutorial author performing the the task. Both forms of instructional content have strengths and weaknesses. Static tutorials are easy to scan forward and backward because they show all instructions. Offering both text and images, they are well suited for people who prefer to learn by looking at images and those who prefer to learn by reading text \cite{Harrison:1995uh}. However, it can be difficult for users to understand continuous, complex manipulations such as painting a region, adjusting control points, or rotating a 3D object in static tutorials. In contrast, videos are effective at showing exactly how an application responds to user interaction, but it is hard to navigate back to previous steps or to look ahead in a video timeline \cite{Pongnumkul:2011ju}.

We hypothesize that a combination of static and video instructions can improve users’ success in following tutorials. We focus on image-editing software in particular, because it is widely used and has a large collection of tutorials accessible in bookstores (books and magazines) and on the web (e.g., user forums and video sharing sites), but we suspect that our findings are generally applicable to visual editing software. With mixed static and video tutorials, users may effectively learn complicated actions (e.g., applying brush strokes) from tutorial video clips, and quickly access basic actions (e.g., copying a layer) from static text and images.

To test our hypothesis, we carried out a within-subjects study comparing static, video, and mixed media tutorials. 12 participants completed three workflows, one for each format. We found that videos are especially valuable for actions that involve brushing, control point manipulation, and adjustment of continuous parameters. We also found that the availability of video reduces the number of repeated attempts users make to execute a step. The study results led to four design guidelines for mixed media tutorials: 1) offer a scannable overview of steps; 2) include small but legible videos; 3) add visualizations of canvas interactions such as brushing to the videos, and 4) enable users to choose the most appropriate visual representation for each step.

To enable instructors to create mixed media tutorials, we introduce MixT, a system that takes a user demonstration and automatically generates mixed tutorials that show static step-by-step content and also include in-place video clips for each operation (Figure~\ref{fig:mixt_teaser}). MixT generates these materials from screencapture video and recorded traces of application commands and input device events. MixT segments video into steps, applies video compositing techniques to focus on salient screen regions, and highlights canvas interactions through mouse trails. The web-based tutorials give users interactive control over when to see images or videos, and how to render videos. A quantitative analysis of nine automatically generated MixT tutorials indicates that our algorithms for segmenting videos into steps and detecting salient regions within frames are effective ({\textless}8\% error rates). In addition, informal user feedback suggests that MixT tutorials were as effective as manually created tutorials in helping users complete tasks.

In summary, the main contributions of this paper include:

\begin{itemize}
  \itemsep -2pt
  \item a categorization of the types of user operations for which video is useful.
  \item a set of design principles for how to embed video in step-by-step tutorials, derived from a formative study.
  \item a general approach for automatically generating mixed media tutorials from demonstrations, and algorithms for implementing this approach for Adobe Photoshop.
  \item an evaluation of automatically generated mixed media tutorials.
\end{itemize}
