%!TEX root = ../thesis.tex
\section{Conclusion}

This chapter introduced MixT, a system that automatically generates step-by-step mixed media tutorials from user demonstrations. We motivated the design of MixT through a formative study that suggested that step videos help users understand complex direct manipulation operations. MixT’s architecture uses a command log, an input device log, and screencapture video to generate tutorials. It applies video compositing techniques to focus on salient information, and highlights interactions through mouse trails. Our informal evaluation suggests that automatically generated MixT tutorials were as effective in helping users complete tasks as tutorials that were created manually.

% \section{Limitation and Future Work}

The current MixT implementation has some important limitations that should be addressed in future work. One missing yet interesting component is the audio content, such as a tutorial author's narration in the video demonstrations. Spoken explanations of the demonstrated actions can help viewers understand the rationale behind a sequence of steps. However, narration and interactions may not always occur in synchrony and it is an open problem to segment combined audio and video tracks appropriately into steps. MixT also does not provide opportunities for the tutorial creator to edit a demonstration. To maximize the benefits of mixed media tutorials, we are interested in exploring ways to provide an editing interface for tutorial authors to easily examine and modify automatic results, and to add annotations that can provide rationale in a lightweight way before sharing their demonstrations.
