%!TEX root = thesis.tex
\begin{abstract}

When attempting to accomplish unfamiliar tasks, people often look for online tutorials to follow instructions. However, producing high-quality instructions requires authoring expertise and a significant time investment in editing; navigating a tutorial using existing tools remains inefficient for following step-by-step instructions.

In this thesis, I introduce video-based recording, editing, and playback tools optimized for creating and consuming interactive tutorials from author demonstrations. Using video and audio analysis techniques, my goal is to dramatically reduce the effort required to produce high-quality instructions for amateur creators. By designing new tutorial formats combined with video content, my tools in turn improve viewers' learning experience.

I present a series of systems that create effective tutorials to support this vision. My approaches acquire videos and high-level events that are important to a learner.
%
Information is derived from either software event logs or user annotations of physical activities using our authoring tools.
%
By combining the input streams, I design algorithms that automatically generate novel instructional formats, which can be edited by authors:
%
1) \keyword{MixT} is a system that automatically generates step-by-step mixed media tutorials from software demonstrations.
%
2) \keyword{DemoWiz} is a tool that provides an increased awareness of upcoming events in a software application demonstration.
%
3) \keyword{DemoCut} is a semi-automatic video editing tool for physical tasks.
%
4) \keyword{Kinectograph} is a recording device that automatically follows an instructor for filming a physical demonstration.
%
5) \keyword{DemoDraw} is a multimodal system to generate step-by-step illustrations from human movement demonstration.

All together, these tools enable amateur authors to create effective tutorials. My systems are evaluated by a series of studies, from which I describe the significance and challenges of my video-based methods.

\end{abstract}
