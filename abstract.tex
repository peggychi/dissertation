%!TEX root = thesis.tex
\begin{abstract}

When attempting to accomplish unfamiliar tasks, people often look for tutorials to follow instructions. While it is easy to access online instructions shared by domain experts, navigating step-by-step guidance using existing tools remains inefficient. In addition, producing high-quality instructions that are easy to follow requires authoring expertise and a significant time investment in editing.
%
This dissertation introduces video-based recording, editing, and playback tools optimized for creating and consuming tutorials from author demonstrations. Our interactive systems capture videos and high-level events that are important to a learner. Using video and audio analysis techniques, we develop algorithms that automatically produce high-quality instructions, which dramatically reduce the effort required for amateur creators. By introducing novel tutorial formats combined with video content, these designs in turn improve viewers' learning experience.

We present a series of authoring tools that enable amateur authors to create effective tutorials:
1) \keyword{MixT} is a system that automatically generates step-by-step mixed media tutorials from software demonstrations.
%
2) \keyword{DemoWiz} is a tool that provides an increased awareness of upcoming events in a software demonstration video.
%
3) \keyword{DemoCut} is a semi-automatic video editing tool for physical tasks.
%
4) \keyword{Kinectograph} is a recording device that automatically follows an instructor for filming a physical demonstration.
%
5) \keyword{DemoDraw} is a multi-modal system to generate step-by-step motion illustrations from author's body movements.
%
Current authoring practices from professionals are encoded into automatic algorithms and interactive techniques. These systems are evaluated through a series of studies, which demonstrate that users can efficiently create and follow concise instructions using our tools.

% in full text:
% When attempting to accomplish unfamiliar tasks, people often look for tutorials to follow instructions. While it is easy to access online instructions shared by domain experts, navigating step-by-step guidance using existing tools remains inefficient. In addition, producing high-quality instructions that are easy to follow requires authoring expertise and a significant time investment in editing. This dissertation introduces video-based recording, editing, and playback tools optimized for creating and consuming tutorials from author demonstrations. Our interactive systems capture videos and high-level events that are important to a learner. Using video and audio analysis techniques, we develop algorithms that automatically produce high-quality instructions, which dramatically reduce the effort required for amateur creators. By introducing novel tutorial formats combined with video content, these designs in turn improve viewers’ learning experience.
% We present a series of authoring tools that enable amateur authors to create effective tutorials: 1) MixT is a system that automatically generates step-by-step mixed media tutorials from software demonstrations. 2) DemoWiz is a tool that provides an increased awareness of upcoming events in a software demonstration video. 3) DemoCut is a semi-automatic video editing tool for physical tasks. 4) Kinectograph is a recording device that automatically follows an instructor for filming a physical demonstration. 5) DemoDraw is a multi-modal system to generate step-by-step motion illustrations from author’s body movements. Current authoring practices from professionals are encoded into automatic algorithms and interactive techniques. These systems are evaluated through a series of studies, which demonstrate that users can efficiently create and follow concise instructions using our tools.

\end{abstract}
