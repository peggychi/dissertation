%!TEX root = ../thesis.tex
\section{Conclusion}

In this chapter, we presented DemoCut, a semi-automatic video editing
system that helps users create clear and concise video tutorials of
DIY tasks. The key idea behind our approach is to combine rough user
annotations with simple video and audio analysis techniques in order
to segment the input recording and apply appropriate editing
effects. Our small user evaluation suggests that video authors are
able to create effective video tutorials using DemoCut, and the
qualitative feedback includes encouraging positive reactions to the
annotation and editing workflow, as well as the automatic editing
effects.

% \section{Limitations and Future Work}

Our implementation is based on several simplifying assumptions that
limit generality. We assume a single, static camera position that
shows all relevant actions and a quiet indoor environment with
constant lighting and little background noise. In order to detect static shots  that should be skipped, our video analysis assumes a static background. Our audio analysis assumes that all non-silent sections of audio are narration, but this may not always be the case. Loud non-speech sounds, such as chopping or the sound of a sewing machine, can lead to errors in our editing effect decisions.

As was pointed out by several of our study participants, making effect decisions individually for each segment can lead to inconsistencies in playback speed as the video transitions from segment to segment. A more global approach that looks at all video effects together and enforces  smooth transitions between adjacent segments would help address some of these artifacts.
%
In addition to addressing these limitations, we see several promising
directions for future work.

\subsubTitleBold{Multiple camera footage.} We designed DemoCut to work with
footage from a single, static camera. One interesting avenue for
future work is to consider footage from multiple cameras. Prior work has compared different camera views capturing physical
tasks for remote collaboration \cite{Fussell:2003te,Ranjan:2007}. Similarly, DemoCut could try to automatically select the best view for
each segment based on user annotations as well as the video content
(e.g., choosing a zoomed view for closeups, switching to a
different view when there are occlusions). % \bh{cite Ranjan here?}

\subsubTitleBold{Support viewer's learning.} In this work, we focus on producing
well-edited video tutorials. However, we could also imagine generating
different output formats, including indexed
videos, step-by-step instructions, or mixed media tutorials, similar
to those presented by Chi et al.~\cite{Chi:2012:MAG:2380116.2380130}. Another natural extension would
be to develop interactive components that monitor user actions and
provide realtime guidance and feedback for general DIY tasks. Follow-up studies to understand viewer's learning experience would be useful for refining the automatic editing effects and interactive design.
% Different output formats and understand viewer's perspective.
% {\bf Interactive DIY tutorials.} Recent work has demonstrated
% interactive tutorial systems that help users follow specific types of
% physical tasks, such as assembling Duplo models~\cite{Gupta:2012ku}.
% A natural extension of our work would
% be to develop interactive components that monitor user actions and
% provide realtime guidance and feedback for general DIY tasks.

\subsubTitleBold{Generalize to other instructional video domains.} One exciting direction is to explore other areas where our techniques could be applied, such as software learning, music instruction, and video lectures. Each domain may require slightly different analysis and segmentation rules. For example, the system could use a log of executed operations to adjust segment boundaries for software tutorials, or incorporate pitch detection when analyzing music instruction.

% \section{Acknowledgments}

% Work at Berkeley was supported by Adobe and a Berkeley Fellowship for Graduate Studies.
% %
% We thank the YouTube users (in alphabetical order) \textit{donyboy73, Griffin Hammond on Indy Mogul, John NYCCNC, Matt Richardson on MAKE, mjfpieters, and TheMuskokaPainter} for sharing their insights on DIY tutorials in our interviews.

%\bh{still need to fix widows and orphans throughout.}
