%!TEX root = ../thesis.tex

\subsection {Discussion}

%We discuss general findings from the 3 studies.
% \dan{Make this a discussion about all 3 studies together, what was learned overall? Argue that DemoDraw is good.}
%
Participants were clearly excited about the overall experience using both the \phaseI{} and the \phaseII{}.
%Comments we received included, \systemname{} is a \iquote{REALLY COOL SYSTEM} (Study 2-P7), \iquote{Seriously, the poses look really cool! I want the last one to be my profile picture} (Study 2-P3), and \iquote{Great idea overall, this will save lots of time for whoever is designing these artworks :)} (Study 3-P3).
%
Some explicitly pointed out their enjoyment when using our system:
% \iquote{Overall, it's a lot of fun, and I can see it being really helpful for anybody trying to explain motion} (P3),
\iquote{it accurately captures how much fun I had making it. :)} (Study 2-P9), and
\iquote{For professional artists, the system not only increases their productivity, but also brings joy and fun to this kind of tasks} (Study 2-P10).

Participant feedback suggests motion illustrations generated by \systemname{} are expressive enough to depict their demonstrations. Our multi-modal interface with our motion analysis and rendering algorithms enabled users to quickly create step-by-step diagrams. Recall that current methods using existing software tools to create similar diagrams take significant time and making visual or spatial changes is difficult.

\bjoern{I'm not really sold on the rest of this section - some of it feels too low level. Not sure what to do here.}

\dan{One of the main things to discuss (with some convincing arguments based on specific results from the three studies) is whether we achieved our primary goals and answered research questions:
1) the system makes good looking diagrams quickly; (reference back to how long current methods take)
2) the iterative/interactive demonstration style is key to this success. (reference back to how non-iterative current methods are) }

\dan{Another discussion topic is the implications of this system. If the iterative/interactive demonstration style worked here, would it be better in other demonstration systems too? What should interaction designers learn from this at a level above the specific motion illustration task?}

\subsubsection{Illustration Styles}
In Study 1, motion arrows successfully conveyed the majority of movements. For some movements when arrows failed to express the intent, other illustration effects might clarify the details to depict the start, intermediate, and end poses. For example, in the first motion set of Study 1, the circular hand movements and squatting action of Step 3 might not be easily interpreted (see Table~\ref{tab:study1_errors}), but for the same motion, participants preferred stroboscopic effect that clearly showed the arm movements and transition in height (see Figure~\ref{fig:study3_effects}).

Furthermore, as \systemname{} captures the continuous motion sequence in 3D from a demonstrator, our system also generates animations showing the dynamic movements.
%
In the warm-up task of Study 2 that captured the second motion set in Study 1, some participants explained that the playback animation of the recording clarified the motion where they incorrectly interpreted the start position.
%
We propose that as motion arrows can efficiently and effectively express most of the motions, a mixed-media version can be created that has been shown to be useful for clarifying step-by-step instructions \cite{chi2012mixt}, where viewers can selectively review part of a static diagram with in-place animation playback. In addition, the 3D reconstruction also makes it possible to review motions from different angles.
%
All in all, our technology enables both instructors and viewers to interactively create and review motion illustrations in multiple ways.
