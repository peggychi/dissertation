%!TEX root = main.tex

\section{Related Work}
Our work is related to research in demonstration-based authoring and motion visualization techniques.

\subsection{Demonstration-Based Authoring}

User demonstrations have been harnessed to generate explanatory, educational or entertainment media in domains including software tutorials~\cite{Bergman:2005:DocWizards,Grabler:2009jj}, animation~\cite{Barnes:2008:VideoPuppetry,held20123d}, 3D modeling~\cite{Zhang:2013:BodyAvatar}, or physical therapy~\cite{Yeager:EECS-2013-91}.
%
For many of these systems, captured demonstrations are treated as fixed inputs that are then processed using fully or semi-automated techniques to produce a visualization.
%
Work that falls into this category includes: generating step-by-step software tutorials from video or screen recordings with DocWizards~\cite{Bergman:2005:DocWizards}, Grabler \ea's system~\cite{Grabler:2009jj}, and MixT~\cite{Chi:2012:MAG:2380116.2380130}, and automatically editing and annotating existing video tutorials with DemoCut~\cite{Chi:2013:DGC:2501988.2502052}.
%
This workflow is similar to graphics research that transforms existing artifacts into illustrations or animations.
Examples include: using technical diagrams to generate exploded views~\cite{li2008automated}, mechanical motion illustrations~\cite{mitra2010illustrating}, or Augmented Reality 3D animations~\cite{Mohr:2015:RTD:2702123.2702490}; using short videos to generate storyboards~\cite{goldman2006schematic}; creating assembly instructions by tracking 3D movements of blocks in DuploTrack~\cite{Gupta2012DuploTrack}; and closely related to our work, using existing datasets of pre-recorded motion capture sequences to generate human motion visualizations with systems by Assa \ea~\cite{assa2005action,assa2008motion}, Choi \ea~\cite{choi2012retrieval}, and Bouvier-Zappa \ea~\cite{bouvier2007motion}.

Animation is one domain where demonstration is often incorporated into the authoring worfklow in a more interactive manner. For example, GENESYS~\cite{Baecker:1969:GENESYS}, one of the earliest computer animation systems, allows users to perform motion trajectories and the timing of specific events with sketching and tapping interactions. Performance-based animation authoring remains a common approach, and recent work shows how physical props can be incorporated to support layered multi-take performances ~\cite{Dontcheva:2003:LAC:1201775.882285,Gupta:2014:MotionMontage} and puppetry~\cite{Barnes:2008:VideoPuppetry,held20123d}.

While the primary goal of performance-based animation systems is to accurately track and re-target prop motions to virtual characters, \systemname{} focuses on the mapping from recorded body movement demonstrations to static illustrations conveying those motions.
Some previous systems have also mapped body movement to static media:
BodyAvatar~\cite{Zhang:2013:BodyAvatar} treats the body as a proxy and reference frame for ``first-person'' body gestures to shape a 3D avatar model and
a Manga comic maker~\cite{lumb_manga_2013} maps the body pose directly into a comic panel.
%and Tweetris~\cite{Freeman:2013:Tweetris} where body shape is used to directly select puzzle shapes by shape similarity.
Systems using interactive guidance for teaching body motions are essentially the inverse of \systemname{}. Examples include YouMove~\cite{anderson2013youmove} that teaches moves like dance and yoga, and Physio@Home~\cite{Tang:2014:Physio@Home} that guides therapeutic exercises.

\subsection{Motion Visualization}

Several of the systems above focus on developing automated algorithms to visualize various dynamic behaviors, such as mechanical motion~\cite{li2008automated,mitra2010illustrating,Seligmann:1991:AGI:127719.122732}, motion in film~\cite{goldman2006schematic}, molecular flexibility~\cite{Bryden-TVCG2012}, and human movements~\cite{assa2005action,bouvier2007motion,choi2012retrieval}.
%
Much of this work is inspired by formalizing techniques and principles for hand-crafted illustrations~\cite{Agrawala:2011:DPV:1924421.1924439}.
% We explore such principles for body movement diagrams in the following section.
% Much of this work is inspired by existing hand-crafted illustrations\wil{Refs from DanG's work, Macaulay}, instructional texts on depicting motion (\wil{Ref Scott McCloud and others}), and cognitive psychology findings that suggest how humans construct mental models of moving objects (\wil{Ref Hegarty, Tversky}).
%
Bouvier-Zappa et al.'s~\cite{bouvier2007motion} automatic approach visualizes large collections of pre-recorded motion capture sequences.
We support many of the same visualization techniques, including motion arrows, overlaid ghosted views, and sequences of poses, but we introduce an interactive approach for authors to create illustrations for particular motions to share with others.
%
Since such demonstrations often involve mistakes and repeated takes of the motion, \systemname{} supports interactions to help authors review and retake portions of their demonstrations.
Moreover, the interactive nature of \systemname{} enables more fine-grained controls for adjusting visualization parameters and compensating for idiosyncratic characteristics of automated algorithms.
